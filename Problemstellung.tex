\chapter{Problemstellung}

Das Robot Operating System (im folgenden ROS genannt), ist ein Framework mit Schwerpunkt auf der Entwicklung von Roboter-Software. Die Entwicklung begann 2007 und wird heute größtenteils von Willow Garage unter Open Source Lizenz weitergeführt. ROS versteht sich selbst als Ökosystem, das unterschiedlichste Robotertypen mit verschiedener Konfiguration unterstützten möchte. Durch sein modularen Aufbau ist es flexibel genug sowohl in der Service-Robotik als auch bei Industrierobotern Anwendung zu finden. Unter ROS wird jeder Bestandteil eines Robotersystems (Navigation, Bilderkennung, Armregelung, etc.) getrennt und kann daher unabhängig von einander entwickelt werden. Eine solche Funktionseinheit wird in ROS als Paket bezeichnet. Mit diesem Ansatz ist es möglich, sich für unterschiedliche Roboter aus dem großen Angebot vorhandener Pakete die jeweils benötigten zusammenzustellen und somit eine individuelle Roboter-Konfiguration aufzubauen. ROS selbst dienst dabei als Schnittstelle zwischen den verschiendenen Modulen. Die dadurch entstehende Flexibilität spiegelt sich in der langen Liste von 41 Roboter\cite{ros_robot_list}, die erfolgreich unter ROS betrieben werden, wieder.

Diese Freiheit hat einen großen Teil zur weiten Verbreitung von ROS beigetragen. Was auf der einen Seite Grundpfeiler des Erfolges ist, setzt auf der anderen Seite ein tiefes Verständnis der Komponenten und deren Wechselwirkungen voraus. So gibt es beispielsweiße eine Vielzahl verschiedener Navigationsalgorithmen, die, abhängig von Roboter und Einsatzszenario, unterschiedlich effizient das Navigationsproblem lösen. Aber auch für die Entwickler von Komponenten gibt es Probleme zu lösen. So sind die Entwickler in der Regel bemüht, ihre Komponenten kompatibel zu verschiedenen Robotermodellen zu entwickeln. Um dies zu gewährleisten müssen die Entwickler für jede neue Version ihrer Software alle Kombinationen von Komponenten die sie unterstützen möchten in der Simulation oder auf dem echten Roboter testen. 

Erschwerend kommt hinzu, dass ROS selbst auch kontinuierlich weiterentwickelt wird und alle sechs Monate eine neue Version veröffentlicht wird. Das bedeutet, dass sich selbst die API von ROS, die innere Struktur vortlaufend weiterentwickelt und verändert. Resultierend ergeben sich für Entwickler eine große Zahl an Testkonfigurationen, die sie kontinuierlich testen müssen um ein reibungsloses Zusammenspiel sicherzustellen.

An diesem Punkt möchte die aktuelle Arbeit angreifen. Ich unterscheide dabei zwischen zwei verschiedenen Entwickler-Typen und ihren unterschiedlichen Ansprüchen: \begin{itemize}
\item Ein \emph{Applikationsentwickler} plant ein konkretes Szenario, entweder lokal oder in einer Simulation. Ein Szenario besteht dabei häufig, aber nicht zwangsläufig, aus einer Umgebung und einem Roboter. Der Applikationsentwickler möchte in dieser Umgebung eine bestimmte Aufgabe erledigen. Für die Problemlölsung sucht er eine Komponente, die seinen Anforderungen genügt. Diese Anforderungen lassen sich meistens auf physikalische Metriken zurückführen. So ist beispielsweiße für ein Navigationsszenario die Minimerung der zurückgelegten Fahrstrecke oder der benötigten Zeit wünschenswert. Für eine andere Aufgabenstellung in welcher der Roboter verschiedene Personen mittels Gesichtserkennung identifiziert ist es denkbar, dass der Entwickler eine robuste Komponente einer schenelleren bevorzugt.

\item Ein \emph{Komponentenentwickler} ist selbst Autor einer Komponente und möchte diese kontinuierlich unter den selben Bedingungen testen und validieren. Dadurch erhält er die Bestätigung, dass seine Arbeit zur Qualitätsverbesserung beiträgt. Auch kann es für ihn interssant sein, seine Komponente zusätzlich in weiteren Szenarien zu testen um somit Schwachstellen unter veränderten Bedingungen aufzuzueigen.

\end{itemize}
